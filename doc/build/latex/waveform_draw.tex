%% Generated by Sphinx.
\def\sphinxdocclass{report}
\documentclass[letterpaper,10pt,english]{sphinxmanual}
\ifdefined\pdfpxdimen
   \let\sphinxpxdimen\pdfpxdimen\else\newdimen\sphinxpxdimen
\fi \sphinxpxdimen=.75bp\relax

\PassOptionsToPackage{warn}{textcomp}
\usepackage[utf8]{inputenc}
\ifdefined\DeclareUnicodeCharacter
 \ifdefined\DeclareUnicodeCharacterAsOptional
  \DeclareUnicodeCharacter{"00A0}{\nobreakspace}
  \DeclareUnicodeCharacter{"2500}{\sphinxunichar{2500}}
  \DeclareUnicodeCharacter{"2502}{\sphinxunichar{2502}}
  \DeclareUnicodeCharacter{"2514}{\sphinxunichar{2514}}
  \DeclareUnicodeCharacter{"251C}{\sphinxunichar{251C}}
  \DeclareUnicodeCharacter{"2572}{\textbackslash}
 \else
  \DeclareUnicodeCharacter{00A0}{\nobreakspace}
  \DeclareUnicodeCharacter{2500}{\sphinxunichar{2500}}
  \DeclareUnicodeCharacter{2502}{\sphinxunichar{2502}}
  \DeclareUnicodeCharacter{2514}{\sphinxunichar{2514}}
  \DeclareUnicodeCharacter{251C}{\sphinxunichar{251C}}
  \DeclareUnicodeCharacter{2572}{\textbackslash}
 \fi
\fi
\usepackage{cmap}
\usepackage[T1]{fontenc}
\usepackage{amsmath,amssymb,amstext}
\usepackage{babel}
\usepackage{times}
\usepackage[Bjarne]{fncychap}
\usepackage{sphinx}

\usepackage{geometry}

% Include hyperref last.
\usepackage{hyperref}
% Fix anchor placement for figures with captions.
\usepackage{hypcap}% it must be loaded after hyperref.
% Set up styles of URL: it should be placed after hyperref.
\urlstyle{same}
\addto\captionsenglish{\renewcommand{\contentsname}{Contents:}}

\addto\captionsenglish{\renewcommand{\figurename}{Fig.}}
\addto\captionsenglish{\renewcommand{\tablename}{Table}}
\addto\captionsenglish{\renewcommand{\literalblockname}{Listing}}

\addto\captionsenglish{\renewcommand{\literalblockcontinuedname}{continued from previous page}}
\addto\captionsenglish{\renewcommand{\literalblockcontinuesname}{continues on next page}}

\addto\extrasenglish{\def\pageautorefname{page}}

\setcounter{tocdepth}{0}



\title{waveform\_draw Documentation}
\date{Oct 11, 2018}
\release{}
\author{AVN}
\newcommand{\sphinxlogo}{\vbox{}}
\renewcommand{\releasename}{}
\makeindex

\begin{document}

\maketitle
\sphinxtableofcontents
\phantomsection\label{\detokenize{index::doc}}



\chapter{Intro}
\label{\detokenize{intro::doc}}\label{\detokenize{intro:id1}}\label{\detokenize{intro:welcome-to-waveform-draw-s-documentation}}\label{\detokenize{intro:intro}}

\section{Motivation}
\label{\detokenize{intro:motivation}}
Provide a simple scripted interface to produce timing diagrams to facilitate an
accurate concise representation of logical design intent.


\subsection{Species spotted in the wild}
\label{\detokenize{intro:species-spotted-in-the-wild}}

\begin{savenotes}\sphinxattablestart
\centering
\sphinxcapstartof{table}
\sphinxcaption{Comparisons of similar tools}\label{\detokenize{intro:id2}}
\sphinxaftercaption
\begin{tabular}[t]{|*{3}{\X{1}{3}|}}
\hline
\sphinxstyletheadfamily 
Tool name
&\sphinxstyletheadfamily 
License
&\sphinxstyletheadfamily 
Notes:
\\
\hline
TimingDesigner
&
yes
&\begin{itemize}
\item {} 
Stood the test of time.

\item {} 
Good for datasheet.

\end{itemize}
\\
\hline
TimingEditor
&
No
&\begin{itemize}
\item {} 
GUI based

\end{itemize}
\\
\hline
TimingGen
&
No
&\begin{itemize}
\item {} 
GUI based

\end{itemize}
\\
\hline
TimingAnalyzer
&
No*
&\begin{itemize}
\item {} 
(Java + Jython scripts)

\item {} 
Remarkably Like TimingDesigner

\item {} 
in beta, opensourced

\end{itemize}
\\
\hline
WaveDrom
&
No
&\begin{itemize}
\item {} 
browser based

\item {} 
good for simple diagrams.

\end{itemize}
\\
\hline
tikz-timing
&
No
&\begin{itemize}
\item {} 
Latex Bundle(steep learning), Error prone

\item {} 
extensible with well rendered output

\end{itemize}
\\
\hline
visio
&
yes
&\begin{itemize}
\item {} 
Templated vector editor

\item {} 
Time consuming, high maintenance overhead

\item {} 
Variable Quality of outputs

\end{itemize}
\\
\hline
word/ppt/excel
&
yes
&\begin{itemize}
\item {} 
diagram mode based on visio

\end{itemize}
\\
\hline
vector editors
&
yes/no
&\begin{itemize}
\item {} 
inkscape/omnigraffle etc.

\end{itemize}
\\
\hline
\end{tabular}
\par
\sphinxattableend\end{savenotes}


\subsection{Why another method?}
\label{\detokenize{intro:why-another-method}}
Most tools emulate the familiar look feel and functionality of timing designer.
They are great for producing data sheets or capturing characterized I/O timing,
However they some what lack when it comes to capturing and communication logic
design intend. Advanced features offered by TimingDesigner seem excessive for
pure clock referenced digital design tasks were the meaning is captured best by
way of dependencies, flow graphs and annotations. WaveDrom comes close to this
objective with a simple clean intuitive method to produce consistent renders.
However it soon hits the limit for more complicated waveforms


\subsection{How is waveform\_draw different?}
\label{\detokenize{intro:how-is-waveform-draw-different}}
Waveform\_draw started with the objective of drawing consistent timing diagrams
from a seemingly crude, but quick mark-up description of the logic function in a
manner similar to a value change dump. Under the hood it invokes Latex with
appropriate libraries, and tries to conceal the required heavy latex mark-up
from the user. The diagram is defined with simple intuitive mark-ups capable
of abstracting information typically in a timing diagram, with an easy
template, allowing diagrams to be rendered from a tabular form in Excel. The
original source xlsx is a .zip of several xml files which are interpreted by
excel. So ideally this unzipped version can be maintained under version control
thus avoiding the need for versionaing a binary file. There might even be
automation available within a versioning system suchas Git which could make this
process transparent.


\subsection{Why use Excel or a similar utility?}
\label{\detokenize{intro:why-use-excel-or-a-similar-utility}}
Using such an application provides a simple tabular interface which can often
be intuitive for waveform entry. Further, simple formulas and rudimentary
intelligence provided by *office can be used advantageously to generate a
consistent maintainable and reproducible waveform. A drawing application can be
used, but the main issue with those is maintainability, as well as effort
required.  Waveform\_draw provides a work flow with human readable intermediate
files at many stages. Python and TeX should provide easy extensibility to the
basic feature set:

\fvset{hllines={, ,}}%
\begin{sphinxVerbatim}[commandchars=\\\{\}]
\PYG{o}{.}\PYG{n}{xlsx} \PYG{o}{\PYGZhy{}}\PYG{o}{\PYGZhy{}}\PYG{o}{\PYGZhy{}}\PYG{o}{\PYGZgt{}} \PYG{o}{.}\PYG{n}{csv} \PYG{o}{\PYGZhy{}}\PYG{o}{\PYGZhy{}}\PYG{o}{\PYGZhy{}}\PYG{o}{\PYGZgt{}} \PYG{o}{.}\PYG{n}{tex} \PYG{o}{\PYGZhy{}}\PYG{o}{\PYGZhy{}}\PYG{o}{\PYGZhy{}}\PYG{o}{\PYGZgt{}} \PYG{n}{pdf}\PYG{p}{,}\PYG{n}{svg}\PYG{p}{,}\PYG{n}{png}
      \PYG{p}{(}\PYG{n}{py}\PYG{p}{)}      \PYG{p}{(}\PYG{n}{py}\PYG{p}{)}     \PYG{p}{(}\PYG{n}{TeX}\PYG{p}{)}
\end{sphinxVerbatim}

As an example, the Excel step can be overstepped and a csv generated directly,
for example by parsing a VCD. This can then be back annotated if required. At
the moment this flow is still conceptual but python/tcl/perl scripts could be
deployed to parse plain vcds and provide the required translation. The method
still would not be straightforward as vcd’s capture value change by time units,
not active clock edges.


\subsubsection{In terms of the Dreaded USP}
\label{\detokenize{intro:in-terms-of-the-dreaded-usp}}\begin{itemize}
\item {} 
provides a consitent less fiddly method for annotation.

\item {} 
build on what was lying around in the shed.

\item {} 
a method to deploy tikz-timing within TeX.

\item {} 
nothing proprietary. Implent/improve/enhance some scripting and modify to your hearts content. It was all done in about 2 weeks. The documentation dragged along for  months!

\end{itemize}


\chapter{Getting Started}
\label{\detokenize{intro:getting-started}}

\section{Pre requisites}
\label{\detokenize{intro:pre-requisites}}

\subsection{Excel}
\label{\detokenize{intro:excel}}\begin{itemize}
\item {} 
Some version of Excel capable of saving an XLSX file recommended

\item {} 
Open Office and Libre Office should also work but not exhaustively tested.

\item {} 
Recommended to start from the diagramming template provided within the
package

\end{itemize}


\subsection{python 2,7}
\label{\detokenize{intro:python-2-7}}\begin{itemize}
\item {} 
Current implementation is in python 2.7 , This will be migrated to 3 sometime
in the future.

\item {} 
python requires the following packages
\begin{itemize}
\item {} 
xls2csv

\item {} 
openpyxl

\end{itemize}

\end{itemize}

\fvset{hllines={, ,}}%
\begin{sphinxVerbatim}[commandchars=\\\{\}]
\PYGZdl{} python \PYGZhy{}c \PYG{l+s+s1}{\PYGZsq{}import pkgutil; print(1 if pkgutil.find\PYGZus{}loader(\PYGZdq{}xls2csv\PYGZdq{}) else 0)\PYGZsq{}}
1
\PYGZdl{} python \PYGZhy{}c \PYG{l+s+s1}{\PYGZsq{}import pkgutil; print(1 if pkgutil.find\PYGZus{}loader(\PYGZdq{}openpyxl\PYGZdq{}) else 0)\PYGZsq{}}
1
\end{sphinxVerbatim}


\subsection{Latex}
\label{\detokenize{intro:latex}}\begin{itemize}
\item {} 
TexLive or MikTex for latex. These are usually found bundled in linux
distributions. MikTex can be installed on windows/Mac etc.
\begin{itemize}
\item {} 
texlive/2016 is known good

\end{itemize}

\end{itemize}


\subsection{Get the scripts}
\label{\detokenize{intro:get-the-scripts}}
Clone from

\fvset{hllines={, ,}}%
\begin{sphinxVerbatim}[commandchars=\\\{\}]
\PYGZdl{} git clone /work/scratch3/nairajay/waveform\PYGZus{}draw
\end{sphinxVerbatim}

Snap shot of directory:

\fvset{hllines={, ,}}%
\begin{sphinxVerbatim}[commandchars=\\\{\}]
\PYG{l+m}{7867}  2018\PYGZhy{}04\PYGZhy{}11 17:08 README
\PYG{l+m}{2743}  2018\PYGZhy{}04\PYGZhy{}09 17:01 read\PYGZus{}xlsx\PYGZus{}val.py
\PYG{l+m}{26677} 2018\PYGZhy{}04\PYGZhy{}10 13:28 draw\PYGZus{}wave\PYGZus{}tex.py
\PYG{l+m}{59733} 2018\PYGZhy{}04\PYGZhy{}10 14:19 waveforms\PYGZus{}template.pdf
\PYG{l+m}{27277} 2018\PYGZhy{}04\PYGZhy{}10 12:57 waveforms\PYGZus{}\PYGZus{}template.xlsx
\PYG{l+m}{6071}  2018\PYGZhy{}04\PYGZhy{}10 14:21 run.sh
\end{sphinxVerbatim}


\subsubsection{Versions}
\label{\detokenize{intro:versions}}\begin{description}
\item[{v1.1:}] \leavevmode
Changes to the intermediate CSV to use ‘;’ instead of ‘.’ as the delimiter.
This has the knock on effect that the Annotation type can no longer have B\textbar{}C:c:r
This was anyway not used, so changed to B\textbar{}C:r where B,C still works as chained i
or baseline mode and  :r selects the annotation colour.
:r is the default and cannot be over overridden, it is not Christmas yet!.

\end{description}


\section{Putting it to work}
\label{\detokenize{intro:putting-it-to-work}}\begin{itemize}
\item {} 
Usage with the wrapper script

\item {} 
Understanding Common Errors

\item {} 
MISC set-up info that might be useful

\end{itemize}

tcsh command line:

\fvset{hllines={, ,}}%
\begin{sphinxVerbatim}[commandchars=\\\{\}]
./run.sh \PYGZhy{}wb *\PYGZlt{}workbook.xlsx\PYGZgt{}* \PYG{o}{[} \PYGZhy{}ws *\PYGZlt{}sheet\PYGZus{}name\PYGZgt{}* \PYG{p}{\textbar{}} \PYGZhy{}all \PYG{p}{\textbar{}} \PYGZhy{}active \PYG{o}{]} \PYGZhy{}disp
\end{sphinxVerbatim}
\begin{itemize}
\item {} 
-wb : workbook file name with xlsx extension
\begin{itemize}
\item {} 
-ws : Name of the sheet within the worksheet

\item {} 
-all: All available sheets within the specified workbook

\item {} 
-active : The active worksheet, which is the worksheet in focus when the
file was saved.

\end{itemize}

\item {} 
-disp : display the rendered output with xpdf

\end{itemize}
\begin{enumerate}
\item {} 
Generate a waveform description in the XL file by filling in clocks and
signals.  The description is similar to a value change dump where only a
change/transition needs to be recorded

Details of this can be found embedded in the example
\sphinxstyleemphasis{waveform\_template.xlsx} file. This file has 3 sheets, a file can have as
many sheets as required.  Each sheet can be converted individually or all
sheets can be converted in batch mode as described below.  Sheets can be
exempt by adding the suffix ‘\_nt’ to its name.  If not, when the sheet is
not empty a conversion will be attempted. Failing to comply with the
template some random Tex error will be generated.  Please note tex errors
are difficult to debug.

When converted in batch mode, by \sphinxstylestrong{-all} an option step will collate all the
individual pages into one pdf file which is named \textless{}workbook\_name\textgreater{}.pdf

The recommended approach to start a new waveform is by making a copy of the
template sheet by right clicking on the tab and choosing copy.  Once the copy
is made, and suitably renamed, the contents of the cells in the waveform area,
NOTES. CLK\_MARKS, ANNOTATE can all be cleared (select rows and delete).  The
waveforms\_template sheet can additionally be renamed with \_nt suffix so that it
is never converted but available as reference. This sheet is protected with no
password.  The recommended methods for efficiently creating a waveform is
detailed as cell comments.

Both Microsoft Excel and LibreOfficei can generate a compatible xlsx file.

\item {} 
Save this file. \sphinxstylestrong{Saving is important} as XL will generate values from formulas.
Also the sheet that is in focus at the time of save will become the active
sheet.

\end{enumerate}

\begin{sphinxadmonition}{note}{Note:}
\begin{DUlineblock}{0em}
\item[] \sphinxstylestrong{Q: Can the file be Directly saved to the H drive?}
\item[]
\begin{DUlineblock}{\DUlineblockindent}
\item[] The file can be saved directly to the H drive, However, sometimes Excel
\item[] would report the file to be read only and refuse to save.  In such instances,
\item[] Save with a different name and save as with the old name rectifies the problem
\end{DUlineblock}
\end{DUlineblock}
\end{sphinxadmonition}
\begin{enumerate}
\item {} 
For batch conversions of all sheets from a workbook use

\end{enumerate}

\fvset{hllines={, ,}}%
\begin{sphinxVerbatim}[commandchars=\\\{\}]
./run.sh \PYGZhy{}wb waveforms\PYGZus{}all.xlsx \PYGZhy{}all
\end{sphinxVerbatim}

To convert all sheets that are non empty. A sheet is empty if it has
no valid cell.  A sheet might unintentionally be classified as non empty, in
such case delete the sheet or add \_nt’ to the end of the sheet name. This can
be useful with sheets used to capture additional info.
\begin{enumerate}
\item {} 
Rendering only the active sheets are useful in closing the description-render cycle

\end{enumerate}

\fvset{hllines={, ,}}%
\begin{sphinxVerbatim}[commandchars=\\\{\}]
./run.sh \PYGZhy{}wb waveforms\PYGZus{}all.xlsx \PYGZhy{}active \PYGZhy{}disp
\end{sphinxVerbatim}

or explicitly specifying the -ws sheet\_name

\fvset{hllines={, ,}}%
\begin{sphinxVerbatim}[commandchars=\\\{\}]
./run.sh \PYGZhy{}wb waveforms\PYGZus{}all.xlsx \PYGZhy{}ws \PYGZlt{}sheet\PYGZus{}name\PYGZgt{} \PYGZhy{}disp
\end{sphinxVerbatim}

Use \sphinxstylestrong{-disp}, to open the rendered result. -disp can also be avoided, but a
previously opened pdf reloaded. However, if the pdf is open from windows,
tex will generate an ERROR. Please Refer ERROR section.


\section{Common ERRORS:}
\label{\detokenize{intro:common-errors}}\begin{enumerate}
\item {} 
Nature of Error when the CLK\_MARKS section is enabled but no clock is defined
i.e the clock column is ‘0’ or empty. Ideally this should be the exact copy
of the clock for which the timing cycles are to be drawn, reference in the
cell as =\textless{}cell\_containing\_the\_name\_of\_the\_clk\textgreater{}.

\end{enumerate}

\fvset{hllines={, ,}}%
\begin{sphinxVerbatim}[commandchars=\\\{\}]
Traceback \PYG{o}{(}most recent call last\PYG{o}{)}:
  File \PYG{l+s+s2}{\PYGZdq{}./draw\PYGZus{}wave\PYGZus{}tex.py\PYGZdq{}}, line 565, in \PYGZlt{}module\PYGZgt{}
    \PYG{n+nv}{tex\PYGZus{}blk\PYGZus{}drawedges} \PYG{o}{=} draw\PYGZus{}edge\PYGZus{}lines\PYG{o}{(}signal\PYGZus{}array, clock\PYGZus{}edges,clk\PYGZus{}filter, indent\PYGZus{}level, marked\PYGZus{}edges, tex\PYGZus{}blk\PYGZus{}drawedges\PYG{o}{)}
  ...
  ...
sre\PYGZus{}constants.error: nothing to repeat
ERROR: waveforms\PYGZus{}template.tex convesion failed
\end{sphinxVerbatim}
\begin{enumerate}
\item {} 
Error when the pdf is open by another application, normally from windows.

\end{enumerate}

\fvset{hllines={, ,}}%
\begin{sphinxVerbatim}[commandchars=\\\{\}]
ERROR:!I can\PYGZsq{}t write on file {}`waveforms\PYGZus{}template.pdf\PYGZsq{}.
       (Press Enter to retry, or Control\PYGZhy{}D to exit; default file extension is {}`.pdf\PYGZsq{})
       Please type another file name for output
       ! Emergency stop.
\end{sphinxVerbatim}
\begin{enumerate}
\item {} 
Nature of the error when ‘…’ get replaced with the Unicode equivalent.

\end{enumerate}

\fvset{hllines={, ,}}%
\begin{sphinxVerbatim}[commandchars=\\\{\}]
Traceback (most recent call last):
  File \PYGZdq{}read\PYGZus{}xlsx\PYGZus{}val.py\PYGZdq{}, line 68, in \PYGZlt{}module\PYGZgt{}
    result = convert\PYGZus{}to\PYGZus{}csv(ws\PYGZus{}active)
  File \PYGZdq{}read\PYGZus{}xlsx\PYGZus{}val.py\PYGZdq{}, line 24, in convert\PYGZus{}to\PYGZus{}csv
    csv\PYGZus{}f.writerow([cell.value for cell in row])
UnicodeEncodeError: \PYGZsq{}ascii\PYGZsq{} codec can\PYGZsq{}t encode character u\PYGZsq{}\PYGZbs{}u2026\PYGZsq{} in position 6: ordinal not in range(128)
\end{sphinxVerbatim}


\section{MISC Notes}
\label{\detokenize{intro:misc-notes}}
Script uses the following packages
\begin{itemize}
\item {} 
xls2csv

\item {} 
openpyxl

\end{itemize}

\begin{sphinxadmonition}{note}{Note:}
There is a specific version check for python, at the moment this is hardcoded
to 2.7.10, you may override this in the script.
\end{sphinxadmonition}

The following is needed for xlstocsv conversion from command line

\fvset{hllines={, ,}}%
\begin{sphinxVerbatim}[commandchars=\\\{\}]
mkdir \PYGZhy{}p /home/nairajay/local/lib/python2.6/site\PYGZhy{}packages/
\end{sphinxVerbatim}

The required packages for python may not be available on the host or a
managed system. Python allows mechanisms to install them locally. Creating
virtual env is another option.
With both pip available and access to the outside world, pip\_install \textendash{}user should
suffice for majority of the cases. This should default to \textasciitilde{}/.local/ and
python would search this path by default.

\fvset{hllines={, ,}}%
\begin{sphinxVerbatim}[commandchars=\\\{\}]
mkdir \PYGZhy{}p /home/nairajay/local/lib/python2.6/site\PYGZhy{}packages/

pip\PYGZus{}install \PYGZhy{}\PYGZhy{}user \PYGZlt{}package\PYGZgt{}
\end{sphinxVerbatim}

A messy way is to use easy\_install or using the set-up.py from a tarball. Both
these can lead to problems.

\fvset{hllines={, ,}}%
\begin{sphinxVerbatim}[commandchars=\\\{\}]
\PYG{n+nb}{echo} \PYG{n+nv}{\PYGZdl{}PYTHONPATH}

\PYG{c+c1}{\PYGZsh{} append if not empty}
\PYG{c+c1}{\PYGZsh{} Note: python version specific}
setenv PYTHONPATH /home/\PYGZlt{}user\PYGZgt{}/local/lib/\PYGZlt{}python\PYGZus{}version\PYGZgt{}/site\PYGZhy{}packages
\PYG{c+c1}{\PYGZsh{} run once}
easy\PYGZus{}install \PYGZhy{}\PYGZhy{}prefix\PYG{o}{=}\PYG{n+nv}{\PYGZdl{}HOME}/local xlsx2csv
\end{sphinxVerbatim}

Example commands

\fvset{hllines={, ,}}%
\begin{sphinxVerbatim}[commandchars=\\\{\}]
module use /opt/ipython/modulefiles
module load ipyhton

module load texlive/2016
\PYGZsh{} sometimes it might complain about the tikz\PYGZhy{}timing library, just use what is
\PYGZsh{} available, Seem to work

python ./draw\PYGZus{}wave\PYGZus{}tex.py waveforms\PYGZus{}cancel\PYGZus{}sane.csv waveforms\PYGZus{}cancel\PYGZus{}sane.tex
pdflatex waveforms\PYGZus{}cancel\PYGZus{}sane.tex
pdflatex \PYGZhy{}interaction=nonstopmode waveforms\PYGZus{}cancel\PYGZus{}sane.tex

inkscape \PYGZhy{}z \PYGZhy{}f waveforms\PYGZus{}cancel\PYGZus{}sane.pdf \PYGZhy{}l waveforms\PYGZus{}cancel\PYGZus{}sane.svg

\PYGZsh{} Push button script
./run.sh \PYGZhy{}wb waveforms\PYGZus{}all.xlsx \PYGZhy{}all \PYGZhy{}disp
\PYGZhy{}disp : open xpdf after every render
\PYGZhy{}all  : process all non empty sheets in xlsx, A sheet is considerd non empty if atleast one cell has a value.
        Check for validity of a sheet for parsing to produce waveforms is not considered.
\PYGZhy{}active: render only the active sheet. Along with display can be used for development.
\PYGZlt{}\PYGZhy{}ws sheet\PYGZus{}name\PYGZgt{} : provide the explicit sheet name.

svg: By default svg and png are generated. scg\PYGZsq{}s are generally large files and hence the default dfeature will be turned off in the future.
\end{sphinxVerbatim}

\begin{figure}[htbp]
\centering
\capstart

\noindent\sphinxincludegraphics[scale=0.5]{{xl1}.png}
\caption{The starting excel template.}
\begin{sphinxlegend}
The cell boundaries allow easy segmentation of various areas. This empty
template will not compile or generate a waveform.
\end{sphinxlegend}
\label{\detokenize{intro:id3}}\end{figure}


\chapter{Template and Markup}
\label{\detokenize{walkthrough::doc}}\label{\detokenize{walkthrough:template-and-markup}}\label{\detokenize{walkthrough:walkthrough}}
A worked out example with a recommended flow and other quirks


\section{Organising the .xlsx file}
\label{\detokenize{walkthrough:organising-the-xlsx-file}}
An .xlsx file may contain more than one sheet. The sheets are to be uniquely named.

\begin{sphinxadmonition}{tip}{Tip:}
\sphinxstylestrong{.xlsx} files is a zip archive which include xml descriptions for the various
sheets and some other anciliary information. Although they can be placed under
version control, zips are binary files and hence may not be the best
way to feed git. An unzipped folder might be better suited for version
control
\end{sphinxadmonition}

In addition, we encourage the user to add two sheets named index\_nt and
Notes\_nt. These will not be rendered but is a good mechanism to capture
information regarding the contents of the workbook as well as allow easy
navigation. Navigating between the sheets can be done with the sheets tab, but
maintaining this info within index and providing hyperlinks to sheets with a
short descriptive summary is highly recommended as a method of navigation.

\begin{figure}[htbp]
\centering
\capstart

\noindent\sphinxincludegraphics[scale=0.5]{{settingup1}.PNG}
\caption{\sphinxstylestrong{Fig}:Recomended workbook layout}
\begin{sphinxlegend}
Tabs Index\_nt, Notes\_nt will not be rendered. Tab ‘template’ holds an empty
wafeform template whole tab  ‘waveform\_template’ is a fully working example with
hints added as cell comments.
\end{sphinxlegend}
\label{\detokenize{walkthrough:id1}}\end{figure}


\subsection{Sheet naming convention}
\label{\detokenize{walkthrough:sheet-naming-convention}}\begin{itemize}
\item {} 
index\_nt : for the cover sheet.

\item {} 
notes\_nt : For the end cover sheet, with any additional info

\end{itemize}

\begin{sphinxadmonition}{note}{Note:}
the python parser script will look for the specific character seq \sphinxstylestrong{\_nt}
to ignore sheets that are not to be rendered.
\end{sphinxadmonition}
\begin{itemize}
\item {} 
New sheets can be created by making a copy of template. However, ensure the
sheet name is changed and the number removed.

\item {} 
Keep names short, use the index instead to capture information regarding
content.

\item {} 
Sheet names may or may not be descriptive. For example when using the index
as a navigation device it is perfectly acceptable to names the sheets as
say set1, set2 etc.

\end{itemize}

\begin{sphinxadmonition}{note}{Note:}
Each sheet will be rendered to a pdf, svg or png file with its \sphinxstylestrong{sheet
name}.  When used with \sphinxstylestrong{-all}, in addition to sheets being rendered
individually they will also be bundeled into a single pdf with its name as
the \sphinxstylestrong{workbook} name.
\end{sphinxadmonition}


\section{Working from the template}
\label{\detokenize{walkthrough:working-from-the-template}}
\noindent{\hspace*{\fill}\sphinxincludegraphics[width=640\sphinxpxdimen,height=640\sphinxpxdimen]{{template1}.png}\hspace*{\fill}}

Template is divided into 3 sections row wise.
\begin{itemize}
\item {} 
Waveform description

\item {} 
Notes,

\item {} 
Annotations.

\end{itemize}

The other markers in the template are required by the python parser and is used as a mark-up
Mark-up usually follows the convention :\sphinxstyleemphasis{markup\_name}: , i.e. a reserved mark-up\_name bracketed by colons

\begin{sphinxadmonition}{note}{Note:}
Column 1 or A in the excel sheet cannot be empty. The parser interprets
most markers when placed in the first column. The only exception is :END:
\end{sphinxadmonition}
\begin{quote}\begin{description}
\item[{SCALE}] \leavevmode
The number placed in the next column is used to control the scale of the waveform.
Scale impact how many clocks are rendered and is an important parameter
to fit the required number of clocks after allowing for margins on a ISO:A4
paper in landscape mode.

Default scale is 4, allowing unto 32 clocks (or xl columns) to be rendered

\item[{END}] \leavevmode
When \sphinxstylestrong{:END:} is specified in any column within the same row as \sphinxstylestrong{:SCALE:}, the parser
limits time to that column. See Example TODO

\item[{TITLE}] \leavevmode
Text following title will be placed as the title of the waveform: See example TODO

\item[{NOTE}] \leavevmode
The notes area has three logical columns. The left most column just carries the text NOTE:
This is provided so as to allow placement of multiline notes to form say a bulleted list.

\end{description}\end{quote}


\begin{savenotes}\sphinxattablestart
\centering
\begin{tabulary}{\linewidth}[t]{|T|T|T|}
\hline
\sphinxstyletheadfamily 
:NOTE:
&\sphinxstyletheadfamily 
Marker
&\sphinxstyletheadfamily 
Text
\\
\hline
NOTE:
&
cell\_id\textgreater{}
&
Note accompanying the marker
\\
\hline
NOTE:
&

&
This is a continuation line of note above if ‘Marker’ is empty
\\
\hline
\end{tabulary}
\par
\sphinxattableend\end{savenotes}


\begin{quote}\begin{description}
\item[{CLK\_MARKS}] \leavevmode
Control the edge and the clock on which the clock boundary is drawn.
\begin{itemize}
\item {} 
D:\textbar{}\textbar{} is the marker for edge and

\item {} 
p:1 identifies rising or posedge.

\end{itemize}

\end{description}\end{quote}




\begin{savenotes}\sphinxattablestart
\centering
\begin{tabular}[t]{|*{3}{\X{1}{3}|}}
\hline
\sphinxstyletheadfamily 
:CLK\_MARKS:
&\sphinxstyletheadfamily 
Edge
&\sphinxstyletheadfamily 
Clock
\\
\hline
D:\textbar{}\textbar{}
&
p:1
&\begin{itemize}
\item {} 
name\_of\_the\_clk\_from\_waveform\_section

\item {} 
Do not type name, instead use reference

\item {} 
type ‘=’ in the cell

\item {} 
click on the name of the clock in the waveform window

\end{itemize}
\\
\hline
\end{tabular}
\par
\sphinxattableend\end{savenotes}

\begin{sphinxadmonition}{note}{Note:}
For edges to be drawn it is important that each column apart from any
containing a break ‘\textbar{}’ should be numbered.
\end{sphinxadmonition}
\begin{quote}\begin{description}
\item[{ANNOTATE}] \leavevmode
Mark-up to render edges, constraints etc. Mark-up and link\_type
together decide the intended style of annotation
\begin{description}
\item[{\sphinxstyleemphasis{Annotate} can specify two types of relations}] \leavevmode\begin{itemize}
\item {} 
E:\textless{}\textgreater{} specifies annotation drawn between edges

\item {} 
L:\textless{}\textgreater{} specifies annotation drawn between levels. These can be used to link sampling conditions or combinatorial results.

\item {} 
\textless{}\textgreater{} the exact type of edge and how they are interpreted varies. See table

\end{itemize}
\begin{description}
\item[{Arcs/arrows are drawn with the form \sphinxstylestrong{\textless{}start\_type\textgreater{}-\textless{}endtype\textgreater{}} where start and end are from the list below.}] \leavevmode\begin{itemize}
\item {} 
\textgreater{}  arrow head

\item {} 
*  filled circle

\item {} 
o  small caps ‘o’, open circle

\item {} 
\textbar{}  dimension line

\end{itemize}

\end{description}

Examples: o-\textgreater{}, *-*, *-\textgreater{}, \textbar{}-\textbar{}

\item[{\sphinxstyleemphasis{Link\_type} can take on two values to specify the type of link}] \leavevmode\begin{itemize}
\item {} 
C:r - Specifies a \sphinxstylestrong{C}hained link. i.e a string of arcs connected back to back.

\item {} 
B:r - specified a \sphinxstylestrong{B}aseline link. i.e the arcs all have the same start point, but multiple end points.

\end{itemize}

\item[{\sphinxstyleemphasis{Markers} column is composed of all remaining columns where each column links to a marker.}] \leavevmode\begin{itemize}
\item {} 
For a \sphinxstylestrong{chained (C:r)} link arcs are drawn between pairs of markers, left to right.

\item {} 
For \sphinxstylestrong{baseline (B:r)} links, arcs are drawn between the first specified mark and every successive edge.

\item {} 
When \sphinxstyleemphasis{:ANNOTATE:}  is of type E:\textbar{}-\textbar{}, a constraint is drawn.

\end{itemize}

This is different from the above modes in that it takes exactly two marks and
the next column specifies a text to be placed following the annotation. As an
example, consider annotating access time. Specify an edge of type E:\textbar{}-\textbar{}, C:r
between two markers M1, M2 with text as ‘t\_Acc’. See specific examples in table

\end{description}

\end{description}\end{quote}

\begin{sphinxadmonition}{note}{Note:}
When more than one constraint ends on the same destination, they are drawn
one below other. However, this might cause them to overlap another waveform.
When such conditions are detected the script emits an error of the form::
\sphinxstylestrong{{[}WARNING      554{]}           add\_arrows(): Multiple dimensions drawn may overlap a waveform, Add a spacer in excel if required R24\textgreater{} W24\textgreater{}}
\end{sphinxadmonition}


\begin{savenotes}\sphinxattablestart
\centering
\begin{tabular}[t]{|*{4}{\X{1}{4}|}}
\hline
\sphinxstyletheadfamily 
:ANNOTATE:
&\sphinxstyletheadfamily 
Link Type
&\sphinxstyletheadfamily 
Markers/Action.
&\sphinxstyletheadfamily 
Render
\\
\hline
E:o-\textgreater{}
&
\sphinxstylestrong{C:r}
&\begin{itemize}
\item {} 
C18\textgreater{} C19\textgreater{} C20\textgreater{}

\item {} 
chained edge links

\end{itemize}
&
\sphinxincludegraphics[scale=0.5]{{chained_arrows}.png}
\\
\hline
E:o-\textgreater{}
&
\sphinxstylestrong{B:r}
&\begin{itemize}
\item {} 
C18\textgreater{} C19\textgreater{} C20\textgreater{}

\item {} 
baseline edge links

\end{itemize}
&
\sphinxincludegraphics[scale=0.5]{{baseline_arrows}.png}
\\
\hline
\sphinxstylestrong{L:*-\textgreater{}}
&
\sphinxstylestrong{B:r}
&\begin{itemize}
\item {} 
C18\textgreater{} C19\textgreater{} C20\textgreater{}

\item {} 
basline level links

\item {} 
useful for combinatorial
dependencies, co-sampling
etc

\end{itemize}
&
\sphinxincludegraphics[scale=0.5]{{baseline_level}.png}
\\
\hline
\sphinxstylestrong{L:*-*}
&
\sphinxstylestrong{C:r}
&\begin{itemize}
\item {} 
C18\textgreater{} C19\textgreater{}

\item {} 
two co-samples signals

\end{itemize}
&
\sphinxincludegraphics[scale=0.5]{{chained_sampling}.png}
\\
\hline
\sphinxstylestrong{L:*-*}
&
\sphinxstylestrong{B:r}
&\begin{itemize}
\item {} 
C18\textgreater{} C19\textgreater{}

\item {} 
two co-samples signals

\item {} 
\sphinxstylestrong{preferred method}

\end{itemize}
&
\sphinxincludegraphics[scale=0.5]{{baseline_sampling}.png}
\\
\hline
E:\textbar{}-\textbar{}
&
\sphinxstylestrong{B:r}
&\begin{itemize}
\item {} 
C18\textgreater{} C19\textgreater{} t\_Acc

\item {} 
baseline edge links

\end{itemize}
&
\sphinxincludegraphics[scale=0.5]{{constraint}.png}
\\
\hline
\end{tabular}
\par
\sphinxattableend\end{savenotes}


\section{The empty template}
\label{\detokenize{walkthrough:the-empty-template}}

\begin{savenotes}\sphinxattablestart
\centering
\sphinxcapstartof{table}
\sphinxcaption{Source  \textendash{}\textgreater{}  render}\label{\detokenize{walkthrough:id2}}
\sphinxaftercaption
\begin{tabulary}{\linewidth}[t]{|T|}
\hline

source
\\
\hline
\scalebox{1.000000}{\sphinxincludegraphics[width=600\sphinxpxdimen,height=600\sphinxpxdimen]{{template1}.png}}
\\
\hline
Rendered ouptut
\\
\hline
\sphinxincludegraphics[scale=1.0]{{template1_out}.png}
\\
\hline
\end{tabulary}
\par
\sphinxattableend\end{savenotes}


\chapter{Markup for signal names}
\label{\detokenize{step_by_step::doc}}\label{\detokenize{step_by_step:markup-for-signal-names}}\label{\detokenize{step_by_step:step-by-step}}

\section{Clock}
\label{\detokenize{step_by_step:clock}}\begin{itemize}
\item {} 
add the cycle numbers, not doing so at least till \sphinxstylestrong{:END} leads to errors

\item {} 
Add one clock to the CLK\_MARKS section.

\item {} 
The clock name shall have \sphinxstylestrong{clk\_xxx} in it

\item {} 
Mark-up for clock is \sphinxstylestrong{C\textless{}duty\textgreater{}:} where duty is the duty cycle of the clock.

\item {} 
If the clock name contains \sphinxstylestrong{nclk\_} an inverted clock will be drawn

\end{itemize}


\begin{savenotes}\sphinxattablestart
\centering
\sphinxcapstartof{table}
\sphinxcaption{Step 1}\label{\detokenize{step_by_step:id2}}
\sphinxaftercaption
\begin{tabulary}{\linewidth}[t]{|T|}
\hline

source
\\
\hline
\scalebox{1.000000}{\sphinxincludegraphics[width=600\sphinxpxdimen,height=600\sphinxpxdimen]{{step1_src}.png}}
\\
\hline
Rendered ouptut
\\
\hline
\scalebox{1.000000}{\sphinxincludegraphics[width=600\sphinxpxdimen,height=600\sphinxpxdimen]{{step1_rend}.png}}
\\
\hline
\end{tabulary}
\par
\sphinxattableend\end{savenotes}


\section{Signal}
\label{\detokenize{step_by_step:signal}}\begin{itemize}
\item {} 
Signals have no markers and represent a single bit

\item {} 
The first columns is used for the initial condition, which is 1, 0 or x.

\item {} 
When the first row is empty, parser emits a warning and insert an x

\item {} 
For a transition, place a 1 in the corresponding clock column. This
specifies the state of the signal following the active clock edge at the start
of the cell. i.e the clock edge at its left boundary.

\item {} 
To transition back place a 0 in the immediate clock.

\item {} 
A seq 010 draws a pulse

\item {} 
Only value changes are required to be captured.

\end{itemize}


\begin{savenotes}\sphinxattablestart
\centering
\sphinxcapstartof{table}
\sphinxcaption{Step 2}\label{\detokenize{step_by_step:id3}}
\sphinxaftercaption
\begin{tabulary}{\linewidth}[t]{|T|}
\hline

source
\\
\hline
\scalebox{1.000000}{\sphinxincludegraphics[width=600\sphinxpxdimen,height=600\sphinxpxdimen]{{step2_src}.png}}
\\
\hline
Rendered ouptut
\\
\hline
\scalebox{1.000000}{\sphinxincludegraphics[width=600\sphinxpxdimen,height=600\sphinxpxdimen]{{step2_rend}.png}}
\\
\hline
\end{tabulary}
\par
\sphinxattableend\end{savenotes}


\section{Bus}
\label{\detokenize{step_by_step:bus}}\begin{itemize}
\item {} 
Buses use the mark-up \sphinxstylestrong{B:\textless{}name\textgreater{}}

\item {} 
Buses take on the additional state ‘u’ in addition to ‘x’

\item {} 
U\textbar{}u is rendered shaded to represent a don’t care value.

\item {} 
x is used interchangeably az HiZ on the bus.

\item {} 
Buses also follow value change representations.

\item {} 
A valid value is interpreted when a non u\textbar{}x character is present by itself in the cell.
\begin{itemize}
\item {} 
Bus values can be coloured with one of four colours by appending {[}c:o\textbar{}r\textbar{}g\textbar{}b{]}

\item {} 
o - orange, r- red …get the picture

\item {} 
use colours sparingly, only if you absolutely need to.

\end{itemize}

\item {} 
characters from the set A-Z a-z0-9\_+-:*() with or without space are supported

\end{itemize}

\begin{sphinxadmonition}{note}{Note:}\begin{itemize}
\item {} 
Reccomended to use atmost 8 char for values of data. More characters may be used but
they tend to overflow the space allocated for a clock when used with scale 4.

\end{itemize}
\end{sphinxadmonition}

\begin{sphinxadmonition}{caution}{Caution:}
While colours are provided their grey scale weights have not been chosen appropriately. Thus use of colours may be confusing in grey scale prints.
\end{sphinxadmonition}

In the example below, bus illustrates use of colouring.
Bus2 has no initial condition defined and the annotation although starts from cell C13 overflows into adjacent cells within excel. Since there is no value change in subsequent cells they are rendered cleanly. The example demonstrates characters that can be used for the annotation, as well as the use of x to render Hiz


\begin{savenotes}\sphinxattablestart
\centering
\sphinxcapstartof{table}
\sphinxcaption{Step 3}\label{\detokenize{step_by_step:id4}}
\sphinxaftercaption
\begin{tabulary}{\linewidth}[t]{|T|}
\hline

source
\\
\hline
\scalebox{1.000000}{\sphinxincludegraphics[width=600\sphinxpxdimen,height=600\sphinxpxdimen]{{step3_src}.png}}
\\
\hline
Rendered ouptut
\\
\hline
\scalebox{1.000000}{\sphinxincludegraphics[width=600\sphinxpxdimen,height=600\sphinxpxdimen]{{step3_rend}.png}}
\\
\hline
\end{tabulary}
\par
\sphinxattableend\end{savenotes}


\subsection{Breaks}
\label{\detokenize{step_by_step:breaks}}
The text decoration on a signal name can be additionally controlled by
appending any of the following modifiers to the name.  o \textless{}i\textgreater{} Italics
\begin{quote}

o \textless{}b\textgreater{} Bold
o \textless{}b\textgreater{}\textless{}i\textgreater{} bold italics
\end{quote}

The motivation to include such modifiers is to enhance communication. For
example a signal of interest may be picked out in bold, but a place holder (ie
signal name used is not the exact name in design, or a group of signals used
for capturing design intend) may be optionally marked in italics. As an example
\textless{}addr\_phase\textgreater{} or \textless{}intr\_packet\textgreater{}, which collects all co samples signals which have
the same timing during an address phase, or the output of the interrupt router.
It is often convenient and effective to abstract away trivial detail such as
individual signal names when drafting a conceptual idea.


\section{Other Enhancements}
\label{\detokenize{step_by_step:other-enhancements}}

\subsection{Breaks}
\label{\detokenize{step_by_step:id1}}\begin{itemize}
\item {} 
Waveform breaks may be added by placing a \textbar{} in the columns representing break.

\item {} 
It is recommended that all rows of the column within the waveform window  carry ‘\textbar{}’

\item {} 
Marker rows are automatically exempted. This allows free insertion of marker rows and reducing the chance of parser errors.

\item {} 
While rendering a break, the state before the break is extended across the break. Thus a text annotation on a bus will continue across the break.

\end{itemize}

\begin{sphinxadmonition}{caution}{Caution:}
some gotchas around this yet to be resolved. The parser does not break, but if it does not produce result as expected it is reccomended to flank the break and avoid toggles.
\end{sphinxadmonition}


\subsection{Gated Clock}
\label{\detokenize{step_by_step:gated-clock}}\begin{itemize}
\item {} 
To render a gated clock use G in the cell where the clock is to be gated off.

\item {} 
These mark-ups can also be used to create divided clocks.

\end{itemize}

\begin{sphinxadmonition}{tip}{Tip:}\begin{description}
\item[{Instead of manually placing ‘G’ in cells to create gated clocks, a reccomended method would be to pick a row that lies after the}] \leavevmode\begin{itemize}
\item {} 
as an example  to generate clk\_c below.

\item {} 
=IF((MOD(B51,3)),”G”,B51) will replace all except every 3rd clock with G rendeinrg a div by 3 clock. Row 51 just contains an ascending count

\end{itemize}

\end{description}
\end{sphinxadmonition}


\subsection{Glitches}
\label{\detokenize{step_by_step:glitches}}\begin{itemize}
\item {} 
A mechanism is provided to draw a glitch. A Glitch will be drawn just following the clock edge assuming the Cell contents represents ‘Q’ as opposed to ‘D’ The above is only for interpretation as Q values don’t glitch.

\item {} 
the utility of a glitch is in representing a combo signal, for example  a req that can be taken away w/o an ack. So here the glitch demonstrates intent and behaviour rather than a purely physical signal glitch.

\item {} 
Mark-up used is \sphinxstylestrong{G} by itself on a signal row.

\item {} 
Direction of glitch is inferred from the bounding signal levels. A glitch does not make much sense unless both the pre and post states are logically opposed to the glitch state.

\end{itemize}


\subsection{Combinatorial, Late arriving signals.}
\label{\detokenize{step_by_step:combinatorial-late-arriving-signals}}\begin{itemize}
\item {} 
A class is added to draw a combinatorial or late arriving signal. This provides a mechanism to accurately communicate intent, and a hint as to how the signal is to be implemented. An example of such a signal would be a ready, or ack.

\item {} 
Mark-up for rendering such a transition is to replace 1 with \sphinxstylestrong{0.75} to produce a delayed transition to 1

\item {} 
Mark-up for rendering such a transition is to replace 0 with \sphinxstylestrong{-0.75} to produce a delayed transition to 0

\end{itemize}

\begin{sphinxadmonition}{note}{Note:}
When 0.xx or -0.xx is used to add an uncertainity to a combo signal the actual fraction has no relevance. the transition region always defaults to 25\% of the clock.
\end{sphinxadmonition}

The example below illustrates, glitches, gated clocks(clk\_c), combo uncertainty and breaks


\begin{savenotes}\sphinxattablestart
\centering
\sphinxcapstartof{table}
\sphinxcaption{Step 4}\label{\detokenize{step_by_step:id5}}
\sphinxaftercaption
\begin{tabulary}{\linewidth}[t]{|T|}
\hline

source
\\
\hline
\scalebox{1.000000}{\sphinxincludegraphics[width=600\sphinxpxdimen,height=600\sphinxpxdimen]{{step4_src}.png}}
\\
\hline
Rendered ouptut
\\
\hline
\scalebox{1.000000}{\sphinxincludegraphics[width=600\sphinxpxdimen,height=600\sphinxpxdimen]{{step4_rend}.png}}
\\
\hline
\end{tabulary}
\par
\sphinxattableend\end{savenotes}


\section{Markup for annotations.}
\label{\detokenize{step_by_step:markup-for-annotations}}

\subsection{Creating a markup row.}
\label{\detokenize{step_by_step:creating-a-markup-row}}
In order to add annotations, the parser relies on marked points in the timing
diagram.  The markers can only be placed on a cycle boundary, and this cycle
boundary coincides with the grid on the excel template, which in turn times the
fastest clock.
\begin{itemize}
\item {} 
Markers are created on a special row placed under the signal of interest with
mark-up \sphinxstylestrong{M:} in the signal name column.

\item {} 
The marker points to the edge
flanking the right of the cell where it is placed. This is visually encoded
with ‘\textgreater{}’, as in example.

\item {} 
To place a marker, it is best to use a formula. The formula puts the position
of the cell ie column and row along with \textgreater{}, such as ‘H31\textgreater{}’

\item {} 
=CONCATENATE(SUBSTITUTE(CELL(“address”),”\$”,”“),”\textgreater{}”) will make the naming consistent when row names exceed Z

\end{itemize}

\begin{sphinxadmonition}{note}{Note:}\begin{itemize}
\item {} 
\sphinxstyleemphasis{=CONCATENATE(CHAR(COLUMN()+64)\&ROW(),”\textgreater{}”)}, was the excel formula used previously.

\item {} 
This has been replaced with the robust method above. The simple formula existed in  earlier versions where the scale was fixed and consequently the number of columns was contained. With a relaxed scale the number of columns can grow.

\item {} 
If the note does not appear explicitely marked, change to formula above.

\item {} 
When using the template, adding the mark-up M: will conditionally format the
entire excel row to a hatched pattern. Although the parser does not see or use
this information it is of great benefit at design entry.

\end{itemize}
\end{sphinxadmonition}


\subsection{Annotation.}
\label{\detokenize{step_by_step:annotation}}
Once the markers are placed, they can be used in annotations. Refer section for mark-up.
\begin{description}
\item[{There are two categories of annotations}] \leavevmode\begin{enumerate}
\item {} 
A label with a note to give explicit information a,r,t to a specific location in the waveform.

\item {} 
Relationship and flow patterns.

\end{enumerate}

\end{description}


\subsubsection{Adding a note}
\label{\detokenize{step_by_step:adding-a-note}}

\begin{savenotes}\sphinxattablestart
\centering
\sphinxcapstartof{table}
\sphinxcaption{Step 5}\label{\detokenize{step_by_step:id6}}
\sphinxaftercaption
\begin{tabulary}{\linewidth}[t]{|T|}
\hline

source
\\
\hline
\scalebox{1.000000}{\sphinxincludegraphics[width=600\sphinxpxdimen,height=600\sphinxpxdimen]{{step5_src}.png}}
\\
\hline
Rendered ouptut
\\
\hline
\scalebox{1.000000}{\sphinxincludegraphics[width=600\sphinxpxdimen,height=500\sphinxpxdimen]{{step5_rend}.png}}
\\
\hline
\end{tabulary}
\par
\sphinxattableend\end{savenotes}

The example below illustrates, glitches, gated clocks(clk\_c), combo uncertainty and breaks


\subsubsection{Adding relationsips}
\label{\detokenize{step_by_step:adding-relationsips}}

\begin{savenotes}\sphinxattablestart
\centering
\sphinxcapstartof{table}
\sphinxcaption{Step 6}\label{\detokenize{step_by_step:id7}}
\sphinxaftercaption
\begin{tabulary}{\linewidth}[t]{|T|}
\hline

source
\\
\hline
\scalebox{1.000000}{\sphinxincludegraphics[width=600\sphinxpxdimen,height=600\sphinxpxdimen]{{step6_src}.png}}
\\
\hline
Rendered ouptut
\\
\hline
\scalebox{1.000000}{\sphinxincludegraphics[width=600\sphinxpxdimen,height=500\sphinxpxdimen]{{step6_rend}.png}}
\\
\hline
\end{tabulary}
\par
\sphinxattableend\end{savenotes}


\chapter{Dealing with Asynchronous events}
\label{\detokenize{async::doc}}\label{\detokenize{async:async}}\label{\detokenize{async:dealing-with-asynchronous-events}}
With state of the art digital design, invariable one will run into a
requirement to represent an asynchronous event. Although
asynchronous events in a standard work flow remains strictly controlled there might
exist cases such as interfaces which are to be represented accurately. The
current parser does not provide dedicated support for asynchronous events. However, the
following tricks can be used to implement asynchronous timing events very
effectively. In most cases, the scenario revolves around some kind of clock
domain crossover. Even when the clock is not known using a virtual clock to
time the signal should still work.


\section{Tips for modelling async}
\label{\detokenize{async:tips-for-modelling-async}}\begin{itemize}
\item {} 
Assume we have two async clocks. For modelling purposes, these need not be
accurate relationships. The very idea of representing async clocks is that we
cannot use a predictable mechanism to go from one domain to another. If such
predictable method exist we drop down to a less stringent mechanism for
moving between multi-cycle clocks.

\item {} 
Assume two clocks, say with freq 3 and 5. Use the LCM of these two clocks to
define a master clock.

\item {} 
Create the master clock. This will be rendered but has no relevance apart
from acting as a time base for the clocks of interest ie, 3,5. For the same
reason, we could specify an extremely low duty cycle, to provide something like
an impulse to avoid confusion. for this clock the cycle numbers are done sequentially.

\item {} 
To create the freq5 and freq 3 clocks, we use the idea of gated clocks. ie a form like
1 G G G G 6 etc.

\end{itemize}

\begin{sphinxadmonition}{caution}{Caution:}\begin{itemize}
\item {} 
The clocks rendered cannot achieve 50\% duty cycle. If you wish to do so
the clock has to be drawn using the signal construct. However, doing so makes
maintenance difficult, i.e if we decide to add a cycle inbetween, the entire
signal might need readjusting. It might just be prudent to use the defined method, but add a note to draw
the readers attention to such irregularity.

\item {} 
Using this method implies the pulses are
numberd non sequentially, ie they
assume the clock number from the master clock when active. A new derived clock
construct may be used in the future to overcome this inconsistency.

\end{itemize}
\end{sphinxadmonition}
\begin{itemize}
\item {} 
Rather than hand coding the waveform, a simple trick is to employ an excel
formula such as using MOD() to decide where the clock is active; i.e.
\sphinxstyleemphasis{mod(master\_clk\_cnt,5)==0 then master\_clk\_cnt else G}

\item {} 
repeating for the other clock with mod(master\_clk\_cnt,3) now provides two
async clocks which will suffice for illustration.

\item {} 
At this stage, we could use the master clock to mark edges, However this can
be distracting.  Instead it is recommended to draw edges from both derived
clocks. They will still be rendered in grey solid lines, as not option is
provided yet to control the type of line. This may be extended in the future.

\end{itemize}


\begin{savenotes}\sphinxattablestart
\centering
\sphinxcapstartof{table}
\sphinxcaption{Step 1}\label{\detokenize{async:id1}}
\sphinxaftercaption
\begin{tabulary}{\linewidth}[t]{|T|}
\hline

source
\\
\hline
\scalebox{1.000000}{\sphinxincludegraphics[width=600\sphinxpxdimen,height=600\sphinxpxdimen]{{async_src}.png}}
\\
\hline
Rendered ouptut
\\
\hline
\scalebox{1.000000}{\sphinxincludegraphics[width=600\sphinxpxdimen,height=600\sphinxpxdimen]{{async_rend}.png}}
\\
\hline
\end{tabulary}
\par
\sphinxattableend\end{savenotes}


\chapter{Indices and tables}
\label{\detokenize{index:indices-and-tables}}\begin{itemize}
\item {} 
\DUrole{xref,std,std-ref}{genindex}

\item {} 
\DUrole{xref,std,std-ref}{modindex}

\item {} 
\DUrole{xref,std,std-ref}{search}

\end{itemize}



\renewcommand{\indexname}{Index}
\printindex
\end{document}